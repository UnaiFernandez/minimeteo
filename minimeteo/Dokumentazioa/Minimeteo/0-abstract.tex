\chapter*{\abstract}
\addcontentsline{toc}{chapter}{\abstract}
\setcounter{page}{1}

Proiektu honetan, uneko eguraldiari buruzko hainbat parametroren egoera emango duen estazio meteoroligikoa eraiki da. Fenomeno meteoroligikoen azterketa egiteko hainbat sentsore erabili dira. Horien artean DH11 hezetasun eta temperatura sentsorea, anemometro bat, haizearen abiadura neurtzeko eta plubiometro bat ur jarioarentzako. 

Sentsore desberdinek jasotako datuak \textit{WiFi} bidez transmitituko dira konektatutako gailuetara, ESP8266 wifi modulua erabilita. 

Azkenik, sentsore eta modulu guztien kontrolaz, \textbf{Atmega328p} mikrokontrolagailua arduratuko da. Mikrokontrolagailu hau arduinoak erabiltzen duena da, baina naiz eta arduinoren plataforma oso garatua egon eta liburutegi asko dituen eskuragarri, proiektu honetan mikrokontrolagailua erregistro mailan programatu egin da.

%% hay que alargar y corregir, pero una vez avanzados mas en el proyecto, para poner informacion mas detallada.